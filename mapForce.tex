\documentclass{article}

\usepackage{graphicx}
\usepackage{amsmath,amssymb}
\usepackage{mathtools}

\providecommand{\e}[1]{\ensuremath{\times 10^{#1}}}
\newcommand{\tc}{\ensuremath{T_\mathrm{c}}}
\newcommand{\me}{\ensuremath{\mathrm{e}}}
\newcommand{\perm}{\epsilon_0}
\newcommand{\vect}[1]{\boldsymbol{\mathbf{#1}}}
\DeclareMathOperator{\sgn}{sgn}

\graphicspath{{figs/}}

\begin{document}

\title{Using Frequency Maps as an Excited State Potential Energy Surface}
\author{Steven Strong}
\date{\today}
\maketitle

The excited state potential is a quadratic function of the electric field
\begin{equation}\label{eq:pot}
U = aE_H+bE_H^2,
\end{equation}
where $E_H$ is the electric field at the excited H position $\vect{r}_H$, in the direction of the excited OH bond, $\vect{\hat r}_{OH}$
\begin{equation}
E_H = \vect E(\vect{r}_H) \vect{\cdot} \vect{\hat r}_{OH}.
\end{equation}
Here, boldface denotes vectors and hats denote unit vectors.
The electric field is given by
\begin{equation}
\vect E(\vect r_H) = \sum_j \frac{q_j\vect{r}_{jH}}{r_{jH}^3}
\end{equation}
where $\vect{r}_{jH} = \vect r_H-\vect r_j$, so $E_H$ is
\begin{equation}
E_H = \sum_j \frac{q_j\vect{r}_{jH}\vect\cdot\vect{\hat r}_{OH} }{r_{jH}^3},
\end{equation}
where the sum over $j$ goes over all atoms within sum cutoff distance of the tagged H, except for the atoms on the same water molecule.
The force on particle $\alpha$ due to the potential in eq.~\ref{eq:pot} is 
\begin{align}
\vect F_\alpha =& -\frac{\partial}{\partial \vect r_\alpha} U \\
=& -a\frac{\partial E_H}{\partial \vect r_\alpha} - 2bE_H\frac{\partial E_H}{\partial \vect r_\alpha} .
\end{align}
So, we need to evaluate the derivative $\partial_{ \vect{r}_\alpha} E_H$.
\begin{equation}\label{eq:ehd}
\frac{\partial E_H}{\partial \vect r_\alpha} = \sum_j q_j \frac{\partial}{\partial \vect r_\alpha} \frac{\vect{r}_{jH}\vect\cdot\vect{\hat r}_{OH} }{r_{jH}^3}.
\end{equation}
There are three cases to consider:
\begin{enumerate}
\item $\alpha=j$
\item $\alpha=O$
\item $\alpha=H$
\end{enumerate}
Whenever the atom $\alpha$ is outside the cutoff or is the other $H$ atom on the excited molecule, $\vect F_\alpha = 0$, because the potential is independent of those atom's positions.

\section{Case 1}
We begin with case 1, $\alpha=j$, and focus on the derivative
\begin{equation}
\frac{\partial}{\partial \vect r_j} \frac{\vect{r}_{jH}\vect\cdot\vect{\hat r}_{OH} }{r_{jH}^3}.
\end{equation}
This is a vector derivative of a scalar.
We begin with the $x$-component of this derivative
\begin{equation}\label{eq:xcomp}
\frac{\partial}{\partial x_j} \frac{(x_H-x_j)x_{OH}+y_{jH}y_{OH} + z_{jH}z_{OH}}{|\vect r_H-\vect r_j|^3r_{OH}}. 
\end{equation}
The first term is
\begin{align*}
\frac{\partial}{\partial x_j} \frac{(x_H-x_j)x_{OH}}{|\vect r_H-\vect r_j|^3r_{OH}} =& \frac{x_{OH}}{r_{OH}} \frac{\partial}{\partial x_j} (x_H-x_j)|\vect r_H-\vect r_j|^{-3}\\
=& \frac{x_{OH}}{r_{OH}} \left( \frac{-1}{|\vect r_H-\vect r_j|^3} -3 \frac{(x_H-x_j)}{|\vect r_H-\vect r_j|^4} \frac{\partial}{\partial x_j}|\vect r_H-\vect r_j|\right)  \\
=& \frac{x_{OH}}{r_{OH}} \left( \frac{-1}{|\vect r_H-\vect r_j|^3} +3 \frac{(x_H-x_j)^2}{|\vect r_H-\vect r_j|^5} \right) \\
=& \frac{x_{OH}}{r_{OH}} \left( \frac{-1}{r_{jH}^3} +3 \frac{x_{jH}^2}{r_{jH}^5} \right)
\end{align*}
where we used 
\begin{align}
\frac{\partial}{\partial x_j}|\vect r_H-\vect r_j| =& \frac{\partial}{\partial x_j}\left((x_H-x_j)+(y_H-y_j)^2+(z_H-z_j)^2\right)^{1/2} \\
=& \frac{1}{2|\vect r_H-\vect r_j|}2(x_H-x_j)(-1) \\
=& -\frac{(x_H-x_j)}{|\vect r_H-\vect r_j|}
\end{align}
The second term is 
\begin{align*}
\frac{\partial}{\partial x_j} \frac{y_{jH}y_{OH}}{|\vect r_H-\vect r_j|^3r_{OH}} =&\frac{y_{jH}y_{OH}}{r_{OH}} \frac{\partial}{\partial x_j} |\vect r_H-\vect r_j|^{-3} \\
=& \frac{3y_{jH}y_{OH}}{|\vect r_H-\vect r_j|^4r_{OH}} \frac{\partial}{\partial x_j} |\vect r_H-\vect r_j| \\
=& \frac{3y_{jH}y_{OH}(x_H-x_j)}{|\vect r_H-\vect r_j|^5r_{OH}} \\
=& \frac{3y_{jH}y_{OH}x_{jH}}{r_{jH}^5r_{OH}}
\end{align*}
The third term is identical, with $y$s replaced by $z$s.
The full result of eq.~\ref{eq:xcomp} is
\begin{align*}
\frac{\partial}{\partial x_j} &\frac{x_{jH}x_{OH}+y_{jH}y_{OH} + z_{jH}z_{OH}}{r_{jH}^3r_{OH}} \\
&=\frac{x_{OH}}{r_{OH}} \left( \frac{-1}{r_{jH}^3} + 3 \frac{x_{jH}^2}{r_{jH}^5} \right) + \frac{3y_{jH}y_{OH}x_{jH}}{r_{jH}^5r_{OH}} + \frac{3z_{jH}z_{OH}x_{jH}}{r_{jH}^5r_{OH}} \\
&= \frac{1}{r_{jH}^3} \left[ -\frac{x_{OH}}{r_{OH}} + 3\frac{x_{jH}}{r_{jH}^2r_{OH}}\left( x_{jH}x_{OH} + y_{jH}y_{OH} + z_{jH}z_{OH}\right) \right] \\
&= \frac{1}{r_{jH}^3} \left( -\frac{x_{OH}}{r_{OH}} + 3\frac{x_{jH}}{r_{jH}^2r_{OH}}\vect r_{jH}\vect\cdot\vect r_{OH} \right) \\
&= \frac{1}{r_{jH}^3} \left( -\frac{x_{OH}}{r_{OH}} + 3\frac{x_{jH}}{r_{jH}}\vect{\hat r}_{jH}\vect\cdot\vect{\hat r}_{OH} \right)
\end{align*}
The derivatives with respect to the other two dimensions are analogous, giving
\begin{equation}
\frac{\partial}{\partial \vect r_j} \frac{\vect{r}_{jH}\vect\cdot\vect{\hat r}_{OH} }{r_{jH}^3} = \frac{ 3\vect{\hat r}_{jH}(\vect{\hat r}_{jH}\vect\cdot\vect{\hat r}_{OH}) - \vect{\hat r}_{OH}}{r_{jH}^3}
\end{equation}

Case 1 ($\alpha=j$) corresponds to $\alpha$ within the cutoff but not on the excited molecule.
For this case, plugging the result into eq.~\ref{eq:ehd} gives
\begin{align}
\frac{\partial E_H}{\partial \vect r_\alpha} =& \sum_j q_j \frac{\partial}{\partial \vect r_\alpha} \frac{\vect{r}_{jH}\vect\cdot\vect{\hat r}_{OH} }{r_{jH}^3} \\
=& q_\alpha \frac{ 3\vect{\hat r}_{\alpha H}(\vect{\hat r}_{\alpha H}\vect\cdot\vect{\hat r}_{OH}) - \vect{\hat r}_{OH}}{r_{\alpha H}^3}
\end{align}

\section{Case 2}
We now turn to case 2, $\alpha=O$.
The $x$-component is
\begin{equation}
\frac{\partial}{\partial x_O} \frac{x_{jH}(x_H-x_O)+y_{jH}y_{OH} + z_{jH}z_{OH}}{r_{jH}^3|\vect r_H-\vect r_O|}. 
\end{equation}
The first term is
\begin{align*}
\frac{x_{jH}}{r_{jH}^3}\frac{\partial}{\partial x_O} (x_H-x_O)|\vect r_H-\vect r_O|^{-1} =& \frac{x_{jH}}{r_{jH}^3}\left( \frac{-1}{|\vect r_H-\vect r_O|} - \frac{(x_H-x_O)}{|\vect r_H-\vect r_O|^2}\frac{\partial}{\partial x_O} |\vect r_H-\vect r_O| \right) \\
=& \frac{x_{jH}}{r_{jH}^3}\left( \frac{-1}{|\vect r_H-\vect r_O|} + \frac{(x_H-x_O)^2}{|\vect r_H-\vect r_O|^3} \right) \\
=& \frac{x_{jH}}{r_{jH}^3}\left( \frac{-1}{r_{OH}} + \frac{x_{OH}^2}{r_{OH}^3} \right)
\end{align*}
The second term is
\begin{align}
\frac{y_{jH}y_{OH}}{r_{jH}^3}\frac{\partial}{\partial x_O} |\vect r_H-\vect r_O|^{-1} =& \frac{y_{jH}y_{OH}x_{OH}}{r_{jH}^3r_{OH}^3}.
\end{align}
Again, the third term is analogous.
The full $x$-component is
\begin{align*}
\frac{\partial}{\partial x_O} &\frac{x_{jH}x_{OH}+y_{jH}y_{OH} + z_{jH}z_{OH}}{r_{jH}^3r_{OH}} \\
&= \frac{x_{jH}}{r_{jH}^3}\left( \frac{-1}{r_{OH}} + \frac{x_{OH}^2}{r_{OH}^3} \right) + \frac{y_{jH}y_{OH}x_{OH}}{r_{jH}^3r_{OH}^3} + \frac{z_{jH}z_{OH}x_{OH}}{r_{jH}^3r_{OH}^3} \\
&= \frac{1}{r_{jH}^3r_{OH}}\left( -x_{jH} + \frac{x_{jH}x_{OH}^2}{r_{OH}^2} + \frac{y_{jH}y_{OH}x_{OH}}{r_{OH}^2} + \frac{z_{jH}z_{OH}x_{OH}}{r_{OH}^2} \right) \\
&= \frac{1}{r_{jH}^3r_{OH}}\left[ -x_{jH} + \frac{x_{OH}}{r_{OH}^2} \left( x_{jH}x_{OH} + y_{jH}y_{OH} + z_{jH}z_{OH} \right) \right] \\
&= \frac{1}{r_{jH}^3r_{OH}}\left( -x_{jH} + \frac{x_{OH}}{r_{OH}^2} \vect r_{jH}\vect\cdot\vect r_{OH} \right) \\
&= \frac{1}{r_{jH}^2r_{OH}}\left( -\frac{x_{jH}}{r_{jH}} + \frac{x_{OH}}{r_{OH}} \vect{ \hat r}_{jH}\vect\cdot\vect{\hat r}_{OH} \right)
\end{align*}
So, the result for case 2 in vector form is
\begin{equation}
\frac{\partial}{\partial \vect r_O} \frac{\vect{r}_{jH}\vect\cdot\vect{\hat r}_{OH} }{r_{jH}^3} = \frac{ \vect{\hat r}_{OH}(\vect{\hat r}_{jH}\vect\cdot\vect{\hat r}_{OH}) - \vect{\hat r}_{jH}}{r_{jH}^2r_{OH}}
\end{equation}

Case 2 ($\alpha=O$) is when $\alpha$ is the oxygen atom on the excited molecule.
For this case, plugging the result into eq.~\ref{eq:ehd} gives
\begin{align}
\frac{\partial E_H}{\partial \vect r_O} =& \sum_j q_j \frac{\partial}{\partial \vect r_O} \frac{\vect{r}_{jH}\vect\cdot\vect{\hat r}_{OH} }{r_{jH}^3} \\
=& \sum_j q_j \frac{ \vect{\hat r}_{OH}(\vect{\hat r}_{jH}\vect\cdot\vect{\hat r}_{OH}) - \vect{\hat r}_{jH}}{r_{jH}^2r_{OH}}
\end{align}
Here, the sum over $j$ goes over all atoms within the cutoff, excluding the excited molecule.

\section{Case 3}
We now turn to case 3, $\alpha=H$.
The $x$-component is
\begin{equation}
\frac{\partial}{\partial x_H} \frac{(x_H-x_j)(x_H-x_O)+y_{jH}y_{OH} + z_{jH}z_{OH}}{|\vect r_H-\vect r_j|^3|\vect r_H-\vect r_O|}. 
\end{equation}
The first term is
\begin{align*}
\frac{\partial}{\partial x_H} &(x_H-x_j)(x_H-x_O)|\vect r_H-\vect r_j|^{-3}|\vect r_H-\vect r_O|^{-1} \\
&=  (x_H-x_O)|\vect r_H-\vect r_j|^{-3}|\vect r_H-\vect r_O|^{-1} + (x_H-x_j)\frac{\partial}{\partial x_H} (x_H-x_O)|\vect r_H-\vect r_j|^{-3}|\vect r_H-\vect r_O|^{-1} \\
&= \frac{x_{OH}}{r_{jH}^3r_{OH}} + x_{jH}\left(|\vect r_H-\vect r_j|^{-3}|\vect r_H-\vect r_O|^{-1} + (x_H-x_O)\frac{\partial}{\partial x_H} |\vect r_H-\vect r_j|^{-3}|\vect r_H-\vect r_O|^{-1}\right)\\
&=  \frac{x_{OH}}{r_{jH}^3r_{OH}} + x_{jH}\left[\frac{1}{r_{jH}^3r_{OH}} + x_{OH}\left(|\vect r_H-\vect r_j|^{-3}\frac{\partial}{\partial x_H} |\vect r_H-\vect r_O|^{-1} + |\vect r_H-\vect r_O|^{-1} \frac{\partial}{\partial x_H}|\vect r_H-\vect r_j|^{-3}\right) \right]\\
&= \frac{x_{OH}}{r_{jH}^3r_{OH}} + x_{jH}\left[\frac{1}{r_{jH}^3r_{OH}} + x_{OH}\left(-\frac{x_{OH}}{r_{jH}^3r_{OH}^3} -3\frac{x_{jH}}{r_{OH}r_{jH}^5}\right) \right]   \\
&= \frac{1}{r_{jH}^3r_{OH}} \left[x_{OH} + x_{jH} - x_{jH} x_{OH}\left(\frac{x_{OH}}{r_{OH}^2} +3\frac{x_{jH}}{r_{jH}^2}\right) \right]  \\
\end{align*}

The second term is
\begin{align*}
y_{jH}y_{OH}\frac{\partial}{\partial x_H} |\vect r_H-\vect r_j|^{-3}|\vect r_H-\vect r_O|^{-1} =& y_{jH}y_{OH}\left(-\frac{x_{OH}}{r_{jH}^3r_{OH}^3} -3\frac{x_{jH}}{r_{OH}r_{jH}^5}\right) \\
=&-\frac{y_{jH}y_{OH}}{r_{jH}^3r_{OH}} \left(\frac{x_{OH}}{r_{OH}^2} +3\frac{x_{jH}}{r_{jH}^2} \right)
\end{align*}
where we used the result from the 3rd through 5th lines above.
The full result for the $x$-component is
\begin{align*}
\frac{\partial}{\partial x_H} &\frac{x_{jH}x_{OH}+y_{jH}y_{OH} + z_{jH}z_{OH}}{r_{jH}^3r_{OH}} \\
&= \frac{1}{r_{jH}^3r_{OH}} \left[x_{OH} + x_{jH} - \left(\frac{x_{OH}}{r_{OH}^2} +3\frac{x_{jH}}{r_{jH}^2}\right) \left( x_{jH} x_{OH}+y_{jH}y_{OH} +z_{jH}z_{OH}  \right) \right] \\
&= \frac{1}{r_{jH}^3r_{OH}} \left[x_{OH} + x_{jH} - \left(\frac{x_{OH}}{r_{OH}^2} +3\frac{x_{jH}}{r_{jH}^2}\right) \vect{r}_{jH}\vect\cdot\vect{r}_{OH} \right]
\end{align*}
Putting all the components together, we have
\begin{align*}
\frac{\partial}{\partial \vect r_H} \frac{\vect{r}_{jH}\vect\cdot\vect{\hat r}_{OH} }{r_{jH}^3} =& \frac{1}{r_{jH}^3r_{OH}} \left[\vect{r}_{OH} + \vect{r}_{jH} - \left(\frac{\vect{r}_{OH}}{r_{OH}^2} +3\frac{\vect{r}_{jH}}{r_{jH}^2}\right) \vect{r}_{jH}\vect\cdot\vect{r}_{OH} \right] \\
=& \frac{1}{r_{jH}^2} \left[\frac{\vect{\hat r}_{OH}}{r_{jH}} + \frac{\vect{\hat r}_{jH}}{r_{OH}} - \left(\frac{\vect{\hat r}_{OH}}{r_{OH}} +3\frac{\vect{\hat r}_{jH}}{r_{jH}}\right) \vect{\hat r}_{jH}\vect\cdot\vect{\hat r}_{OH} \right]
\end{align*}

Case 3 ($\alpha=H$) is when $\alpha$ is the excited hydrogen atom.
For this case, plugging the result into eq.~\ref{eq:ehd} gives
\begin{align}
\frac{\partial E_H}{\partial \vect r_H} =& \sum_j q_j \frac{\partial}{\partial \vect r_H} \frac{\vect{r}_{jH}\vect\cdot\vect{\hat r}_{OH} }{r_{jH}^3} \\
=& \sum_j q_j \frac{1}{r_{jH}^2} \left[\frac{\vect{\hat r}_{OH}}{r_{jH}} + \frac{\vect{\hat r}_{jH}}{r_{OH}} - \left(\frac{\vect{\hat r}_{OH}}{r_{OH}} +3\frac{\vect{\hat r}_{jH}}{r_{jH}}\right) \vect{\hat r}_{jH}\vect\cdot\vect{\hat r}_{OH} \right]
\end{align}
Here, the sum over $j$ goes over all atoms within the cutoff, excluding the excited molecule.

\section{Summary}
So, the force on particle $\alpha$ is given by 
\begin{equation}
\vect F_\alpha = -a\frac{\partial E_H}{\partial \vect r_\alpha} - 2bE_H\frac{\partial E_H}{\partial \vect r_\alpha} ,
\end{equation}
where the derivative $\partial_{ \vect r_\alpha} E_H$ is given by
\begin{equation*}
\frac{\partial E_H}{\partial \vect r_\alpha}  = \begin{dcases}
q_\alpha \frac{ 3\vect{\hat r}_{\alpha H}(\vect{\hat r}_{\alpha H}\vect\cdot\vect{\hat r}_{OH}) - \vect{\hat r}_{OH}}{r_{\alpha H}^3} &\alpha=\text{different molecule}\\
\sum_j q_j \frac{ \vect{\hat r}_{OH}(\vect{\hat r}_{jH}\vect\cdot\vect{\hat r}_{OH}) - \vect{\hat r}_{jH}}{r_{jH}^2r_{OH}} &\alpha=\text{excited O} \\
\sum_j q_j \frac{1}{r_{jH}^2} \left[\frac{\vect{\hat r}_{OH}}{r_{jH}} + \frac{\vect{\hat r}_{jH}}{r_{OH}} - \left(\frac{\vect{\hat r}_{OH}}{r_{OH}} +3\frac{\vect{\hat r}_{jH}}{r_{jH}}\right) \vect{\hat r}_{jH}\vect\cdot\vect{\hat r}_{OH} \right] &\alpha=\text{excited H} \\
0 &\alpha=\text{outside cutoff} \\
0 &\alpha=\text{other H on excited molecule}
\end{dcases}
\end{equation*}
Here, the sums over $j$ go over all atoms within the cutoff, excluding the excited molecule.


\end{document}